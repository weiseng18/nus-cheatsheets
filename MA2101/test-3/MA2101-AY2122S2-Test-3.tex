\documentclass[a4paper]{article}
\usepackage[a4paper,margin=0.2in]{geometry}

\setlength{\columnseprule}{0.3pt}

\AtBeginDocument{
\addtolength{\abovedisplayskip}{-1ex}
\addtolength{\abovedisplayshortskip}{-1ex}
\addtolength{\belowdisplayskip}{-1ex}
\addtolength{\belowdisplayshortskip}{-1ex}
}

\pagenumbering{gobble}

% default stuff
\usepackage{amsmath}
\usepackage{amssymb}
\usepackage{enumitem}

% multicolumn package
\usepackage{multicol}

\usepackage{multirow}
\usepackage{makecell}
\renewcommand\theadfont{\bfseries}

\newcolumntype{M}[1]{>{\centering\arraybackslash} m{#1}} % centered m

\newcommand{\abs}[1]{\left\lvert#1\right\rvert}

\newcommand{\ol}[1]{\begin{enumerate}#1\end{enumerate}}
\newcommand{\oll}[1]{\begin{enumerate}[leftmargin=*]#1\end{enumerate}}
\newcommand{\ul}[1]{\begin{itemize}#1\end{itemize}}
\newcommand{\ull}[1]{\begin{itemize}[leftmargin=*]#1\end{itemize}} % no margin

\usepackage{spalign}

\let\mat=\spalignmat
\let\amat=\spalignaugmat

\newcommand{\detmat}[1]{\spaligndelims\vert\vert\spalignmat{#1}\spaligndelims()}

\newcommand{\ro}[1]{\xrightarrow{\begin{subarray}{l}#1\end{subarray}}}
\newcommand{\roTwo}[2]{\xrightarrow{\begin{subarray}{l}#1\\#2\end{subarray}}}
\newcommand{\roThree}[3]{\xrightarrow{\begin{subarray}{l}#1\\#2\\#3\end{subarray}}}

% matrix operations
\newcommand{\rref}{\mathrm{rref}}

\newcommand{\transpose}{^{\text{T}}}
\newcommand{\tr}{^{\text{T}}}

\newcommand{\GJE}{\ro{\text{GJE}}}
\newcommand{\matdim}[1]{\mathrm{dim}(#1)}
\newcommand{\matrank}[1]{\mathrm{rank}(#1)}
\newcommand{\matnullity}[1]{\mathrm{nullity}(#1)}

\newcommand{\minrank}[1]{\mathrm{min}\{#1\}}

\newcommand{\vabs}[1]{\abs{\abs{#1}}}

\newcommand{\by}[2]{$#1 \times #2$}
\newcommand{\laspan}[1]{\mathrm{span}\left\{#1\right\}}

\newcommand{\spc}{\spaligndelims..} % spalign cancel brackets
\newcommand{\spv}{\spaligndelims\vert\vert} % spalign vert
\newcommand{\spr}{\spaligndelims()} % spalign reset

% vector bold
\newcommand{\vb}[1]{\boldsymbol{#1}}
\newcommand{\I}{\vb{I}}
\newcommand{\vz}{\vb{0}}

% expand to row vector
% #1 is the vector, #2 is the length
% refer to MA2001 Tutorial 5 for sample usage
\newcommand{\exprow}[2]{
  \rule[.4ex]{#2}{0.5pt}
  \hspace{0.2cm}
  #1
  \hspace{0.2cm}
  \rule[.4ex]{#2}{0.5pt}
}

% set of column vectors
\newcommand{\vset}[1]{\left\{#1\right\}}

% LA
\newcommand{\spn}[1]{\mathrm{span}(#1)}
\newcommand{\rank}[1]{\mathrm{rank}(#1)}
\newcommand{\nullity}[1]{\mathrm{nullity}(#1)}
\newcommand{\lm}{\lambda}

% matrices
\newcommand{\A}[1][]{\vb{A_{#1}}}
\newcommand{\B}{\vb{B}}
\newcommand{\E}{\vb{E}}
\newcommand{\D}{\vb{D}}
\renewcommand{\P}{\vb{P}}
\newcommand{\Range}{\mathrm{R}}
\newcommand{\Kernel}{\mathrm{Ker}}
\newcommand{\trc}[1]{\mathrm{tr}(#1)}

% vectors
\renewcommand{\u}[1][]{\vb{u_{#1}}}
\renewcommand{\v}[1][]{\vb{v_{#1}}}
\newcommand{\w}[1][]{\vb{w_{#1}}}
\newcommand{\x}[1][]{\vb{x_{#1}}}
\renewcommand{\b}{\vb{b}}
\newcommand{\n}{\vb{n}}
\newcommand{\p}{\vb{p}}


% colors
\usepackage[dvipsnames,table]{xcolor}
\newcommand{\red}[1]{\textcolor{red}{(#1)}}

% shrink
\usepackage{graphicx}

% shorthands
\newcommand{\LVV}{\mathcal{L}(V,V)}
\newcommand{\F}{\mathcal{F}}
\newcommand{\C}{\mathbb{C}}
\newcommand{\R}{\mathbb{R}}
\newcommand{\caseand}{\quad\text{and}\quad}

\newcommand{\inv}{^{-1}}

% reduce spacing before and after headers
\makeatletter
\renewcommand{\section}{
  \@startsection{section}{1}{0pt}{1ex}{1ex}{\normalfont\large\bfseries}}
\renewcommand{\subsection}{
  \@startsection{subsection}{2}{0pt}{1ex}{1ex}{\normalfont\normalsize\bfseries}}
% 5th arg is different for paragraph
\renewcommand{\paragraph}{
  \@startsection{paragraph}{4}{0pt}{1.5ex}{-0.6em}{\normalfont\bfseries}}

\begin{document}
\begin{multicols*}{3}
  \scriptsize
  \section*{\underline{Miscellaneous}}
    \paragraph{0.1 Basis}
      \ull {
        \item The full set $z_i \otimes \zeta^j$ over all $(i, j)$ is a basis for $\LVV$.
        \item The full set $\zeta^i \otimes \zeta^j$ over all $(i, j)$ is a basis for $\mathcal{L}(\hat{V},\hat{V})$. Note that this tensor product only takes in one vector.
      }
    \paragraph{0.2 Trace}
      \ull {
        \item Invariant under similarity transformation (shown in Tut 6 Q6)
        \item Is sum of eigenvalues (shown in Tut 8 Q3)
      }
    \paragraph{0.3 How a matrix describes change} Let $T$ be a 2D LT.
      \[ \mathcal{M}(T) = \mat{a b; c d} \]
      This means that $T(\hat{i}) = \mat{a;c}$ and $T(\hat{j}) = \mat{b;d}$, i.e. the columns tell us how the basic unit vectors change under the transformation.
    \paragraph{0.4 Transpose of LT} Let $\alpha \in \hat{V}$ and let $v \in V$. For each LT $T: V \rightarrow V$, define $\hat{T}: \hat{V} \rightarrow \hat{V}$ as follows:
      \[ \hat{T}(\alpha)(v) = \alpha(Tv) \]
      Then $\hat{T}$ is the transpose of $T$. Since $\hat{T}$ is linear, so $\hat{T} \in \mathcal{L}(\hat{V}, \hat{V})$.
    \paragraph{0.5 Notation}
      \ull {
        \item $I^i_j = 1$ if $i=j$, else 0 (known as Kronecker delta)
        \item Summation over pairs of subscript/superscript
        \item For a matrix $M$, the entry in row $i$ and column $j$ is $M^i_j$.
      }
    \subsection*{Isomorphisms} % Test 1 content
      \paragraph{0.6 Definitions} Let $F$ be a mapping $F:S \rightarrow T$.
        \ull {
          \item Surjection: $\forall t \in T, \; \exists s \in S \quad F(s) = t$
          \item Injection: $\forall s_1, s_2 \in S \quad F(s_1) = F(s_2) \Rightarrow s_1 = s_2$
          \item Bijection: Surjection and Injection
        }
      \paragraph{0.7 Homomorphism} Let $\phi:U \rightarrow V$ be a mapping. It is a homomorphism if
        \begin{align*}
          \phi(u+v) &= \phi(u) + \phi(v) \\
          \phi(au) &= a \phi(u)
        \end{align*}
        If this homomorphism is \underline{also} a bijection, then this is an isomorphism.
      \paragraph{0.8 Note} LI $\Rightarrow$ injective $\Rightarrow$ surjective $\Rightarrow$ bijective $\Rightarrow$ basis
  \section*{\underline{Eigenstuff}}
      \paragraph{4.1 Linear operator} A LT from $V$ to itself
      \paragraph{Complex operator} A linear operator on a vector space over the complex numbers
      \paragraph{4.2 Invariant subspace} A subspace $U \subseteq V$ is invariant wrt $T \in \LVV$ if $Tu \in U, \forall u \in U$.
      \paragraph{4.7 Eigenray} A 1-d invariant subspace. Its elements are eigenvectors.
      \paragraph{4.10 Eigenvector/value} Let $T$ be a linear operator on $V$. A vector $v \in V$ is an eigenvector of $T$ if $Tv = \lm v$ for a particular $\lm \in \F$. $\lm$ is the eigenvalue of this eigenvector.
      \paragraph{4.19 Eigenrays are LI} Consider $N$ eigenrays with different eigenvalues. The set that consists of 1 vector from each eigenray is LI.
        \red{proof by contradiction}
      \paragraph{4.20 Cap on number of eigenvalues} The number of distinct eigenvalues for an operator $V$ cannot be larger than $\dim(V)$. 
        \red{$k$ eigenvalues $\Rightarrow k$ eigenrays $\Rightarrow k$ LI vectors, then show injective linear map from $\F^n$ to $V$}
    \subsection*{FTOA}
      \paragraph{Complex coefficients} Any polynomial with complex coefficients can be completely factorized into the form
        \[ a (x-a_1) (x-a_2) \cdots \]
        where $a, a_i \in \C$. The factorization is unique (excluding reordering)
      \paragraph{Real coefficients} Any polynomial with real coefficients can be expressed in the form
        \[ a (x-a_1) (x-a_2) \cdots (x^2+b_1 x+c_1) (x^2+b_2 x+c_2) \cdots \]
        where $a, a_i, b_i, c_i \in \R$, and where each $b_i^2 < 4c_i$. The factorization is unique (excluding reordering).
      \paragraph{4.21 Existence of an eigenvalue} Every \underline{complex operator} has at least one eigenvalue.
        \paragraph{Proof}
          \oll {
            \item Let $\dim(V) = n$. Let $v \neq 0 \in V$.
            \item The list of vectors $v, Tv, T^2 v, \cdots, T^n v$ contains $n+1$ vectors, so it is not linearly independent.
              \begin{align*}
                c_0 v + c_1 Tv + \cdots + c_n T^n v &= 0 \\
                (c_0 I + c_1 T + \cdots + c_n T^n) v &= 0 & \text{by linearity} \\
                a (T - a_1 I)(T - a_2 I) \cdots v &= 0 & \text{by FTOA}
              \end{align*}
            \item If every $(T - a_j I)$ injective, then $v = 0$, contradiction.
            \item Hence, some $(T - a_j I)$ is not injective, i.e. $(T - a_j I)w = 0$ has infinite solutions.
            \item Hence $a_j$ is an eigenvalue.
          }
    \subsection*{UT matrices}
      \paragraph{4.25 UT} A square matrix $M^i_j$ is upper triangular if $M^i_j = 0$ when $i > j$.
      \paragraph{4.27 Similar} Matrices $A$ and $B$ are similar if $A = P\inv B P$, where $A,B$ are square and of the same size.
      \paragraph{4.27A Similar matrices have the same eigenvalues} Let $M$ be a square matrix, and let $P$ be a change-of-basis matrix. Let $v$ be an eigenvector of $M$, corresponding to eigenvalue $\lm$.
        \begin{align*}
          Mv &= \lm v \\
          P\inv Mv &= \lm P\inv v \\
          P\inv M (PP\inv) v &= \lm P\inv v \\
          (P\inv M P) (P\inv v) &= \lm (P\inv v)
        \end{align*}
        Hence they have the same eigenvalues, but the eigenvectors are now $P\inv v$.
      \paragraph{4.26} Every \underline{complex operator} has a UT matrix. Or, every complex matrix is similar to a UT matrix.
      \paragraph{Idea} Let $N$ be a complex matrix.
        \oll {
          \item Choose arbitrary basis such that the first basis vector is an eigenvector.
            \[ P\inv N P = \mat{\lm_1 \alpha_1 \alpha_2; 0 M^2_2 M^2_3; 0 M^3_2 M^3_3} = \mat{\lm_1 \alpha; 0 M} \]
            where $M$ is 2 by 2 and $\alpha$ is 1 by 2. Note that $\lm_1$ is an eigenvalue of $N$.
          \item Do the same for $M$:
            \[ Q\inv M Q = \mat{\lm_2 \beta; 0 R} \]
            where $\beta, R \in \C$. Note that $\lm_2$ is an eigenvalue of $M$ (also of $N$).
          \item But $M$ is inside $N$, can we claim that we can apply a similarity transformation?
            \begin{equation*}
              \hspace{-\leftmargin}
              \resizebox{\hsize}{!}{
                $\mat{1 0; 0 Q}\inv \mat{\lm_1 \alpha; 0 M} \mat{1 0; 0 Q} = \mat{\lm_1 \alpha Q; 0 Q\inv MQ}$
              }
            \end{equation*}
            so we can concretely apply the similarity transformation by using $\mat{1 0; 0 Q}$.
          \item By 3, we can change the basis of $M$ by applying a similarity transformation to the original matrix. By 4.27A, then $\lm_2$ is indeed an eigenvalue of $N$, so the claim in point 2 is true.
        }
      \paragraph{4.30} The diagonal entries in a UT matrix are its eigenvalues.
      \paragraph{Proof} Let $M$ be a UT matrix.
        \ull {
          \item Consider $M - \lm I$. It is UT, and by definition of $\lm$, $M - \lm I$ must be singular.
          \item Hence, we know one of the diagonal entries are 0, so since we subtracted $\lm$, then one of the diagonal entries of $M$ is $\lm$.
        }
    \subsection*{Diagonal matrices}
      \paragraph{4.31 Diagonal} A square matrix $M^i_j$ is diagonal if $M^i_j = 0$ when $i \neq j$.
      \paragraph{4.30A} Since a diagonal matrix is also UT, its diagonal entries are its eigenvalues.
      \paragraph{4.32-4.34 Diagonalizable} An operator is diagonalizable if
        \ull {
          \item there is a basis wrt which its matrix is diagonal.
          \item or, there is a basis consisting of only eigenvectors
          \item or, the operator has $\dim(V)$ distinct eigenvalues
        }
    \subsection*{Eigenspaces}
      \paragraph{4.35 Eigenspace} Let $T$ be an operator on $V$. Then the eigenspace of $T$ corresponding to eigenvalue $\lm$ is
        \[ E(\lm, T) = Ker(T - \lm I) \]
      \paragraph{4.38 Eigenspaces are mutually disjoint} The sum of eigenspaces corresponding to distinct eigenvalues is a \underline{direct sum}, i.e.the intersection of those eigenspaces is a set containing the zero vector.
      \paragraph{4.39} T is diagonalizable $\iff$
        \[ V = E(\lm_1, T) \oplus E(\lm_2, T) \oplus \cdots E(\lm_k, T) \]
        for distinct eigenvalues $\lm_i$. \red{each eigenspace has a basis of eigenvectors}
    \subsection*{Jordan canonical form}
      \paragraph{Superdiagonal} of a square matrix is the set of elements directly above the diagonal. In the example below, the superdiagonal is exactly the set of all the 1s.
      \paragraph{4.41 Jordan block} A Jordan block of size $m$ is a order $m$ square matrix of the form (e.g. $m=4$):
        \[ \mat{\lm, 1 0 0; 0 \lm, 1 0; 0 0 \lm, 1; 0 0 0 \lm} \]
        where the diagonal entries are a specific $\lm$, the entries along the superdiagonal are 1, and the other entries are 0.
        \\\\
        The Jordan block has only one eigenvector.
      \paragraph{4.42 Jordan basis} A Jordan basis is one such that the matrix of $T$ consists of Jordan blocks:
        \[ \mat{J_1 0 0; 0 J_2 0; 0 0 J_3} \]
        Note that the eigenvalues of this matrix are the diagonal entries in the blocks, since the blocks are UT and so is the matrix.
      \paragraph{4.43} Every \underline{complex operator} has a unique Jordan basis (excluding reordering). \red{real operators may not have, because it might have complex eigenvalues}
      \paragraph{4.44 Jordan canonical form} When a operator is represented by the matrix wrt a Jordan basis, we say that it is in its JCF.
      \paragraph{Note} Real operators also have a JCF, but not as nice.
      \paragraph{Finding change-of-basis matrix for JCF (Tut 7 Q8)} Consider the shearing matrix $S$. Because it is UT we know its eigenvalues, thus we can construct its JCF:
        \[ S = \mat{1 \tan\theta; 0 1} \caseand J = \mat{1 1; 0 1} \]
        From 0.1, we know that the full set $z_i \otimes \zeta^j$ is a basis for $\LVV$, so we can write both matrices in terms of their own bases:
        \begin{align*}
          S &= e_1 \otimes \epsilon^1 + (\tan\theta) e_1 \otimes \epsilon^2 + e_2 \otimes \epsilon^2 \\
          J &= \bar{e}_1 \otimes \bar{\epsilon}^1 + \bar{e}_1 \otimes \bar{\epsilon}^2 + \bar{e}_2 \otimes \bar{\epsilon}^2
        \end{align*}
        Here, the bars represent the new basis under the similarity transformation that creates the JCF.
        \\\\
        Notice that $\tan\theta e_1$ becomes $\bar{e}_1$. Thus, we can define $\bar{e_1} = \tan(\theta)e_1$. Then to ensure that duality is preserved after the transformation, we must have $\bar{\epsilon}^1 = \cot(\theta) \epsilon^1$. Notice we can leave $e_2$ and $\epsilon^2$ as they are. So we have
        \begin{align*}
          \bar{e}_1 &= (\tan\theta) e_1 \\
          \bar{e}_2 &= e_2
        \end{align*}
        and we can represent this in a matrix by 0.3:
        \[ \mat{\tan\theta, 0; 0 1} \]
        and this is indeed the change-of-basis matrix for the JCF.
    \subsection*{Cayley-Hamilton theorem}
      \paragraph{4.45 Multiplicty of an eigenvalue} $\lm$ is
        \ull {
          \item the sum of the sizes of the Jordan blocks corresponding to the eigenvalue
          \item or, the number of times it appears along the diagonal
        }
      \paragraph{Diagonalizability of a matrix} From Tut 7 Q7,
        \ull {
          \item The dimension of an eigenspace is the number of Jordan blocks of an eigenvalue
          \item A matrix is diagonalizable $\iff$ every eigenvalue has an eigenspace has max dimensionality, namely the multiplicity of that eigenvalue.
        }
      \paragraph{4.46 Characteristic polynomial} Let $m_i$ be the multiplicity of eigenvalue $\lm_i$ of a linear operator $T$. Then the characteristic polynomial of $T$ is
        \[ \chi_T(x) = (x - \lm_1)^{m_1} (x - \lm_2)^{m_2} \cdots \]
        Clearly the eigenvalues satisfy $\chi_T(\lm) = 0$.
      \paragraph{Nilpotent matrix} A square matrix $M$ is nilpotent if $M^k = 0$ for some positive integer $k$, known as the index of $M$.
      \paragraph{4.47 Cayley-Hamilton} states that $\chi_T(T) = 0$.
        \paragraph{Proof}
          \oll {
            \item Consider a Jordan block $B = \mat{\lm_1, 1; 0 \lm_1}$, of some larger matrix $M$.
            \item The multiplicity of $\lm_1$ is 2, so there is a $(x-\lm_2)^{m_2}$ term in the characteristic polynomial.
            \item Consider $B-\lm I = \mat{0 1; 0 0}$. Note that it is nilpotent with index 2.
            \item Thus $(B-\lm I)^2 = 0$. This applies to all Jordan blocks. Take the product and we have the Cayley-Hamilton theorem.
          }
      \paragraph{4.50 Finding an inverse} If a matrix $M$ is invertible, then we can keep multiplying $M\inv$ to its characteristic equation until there is a $M\inv$ term, and make $M\inv$ the subject of the formula. In particular, $M\inv$ is a linear combination of $M^i$.
      \paragraph{4.51 Geometric meaning of JCF} Note that in the case with \underline{real nonzero eigenvalues}:
        \resizebox{0.9\hsize}{!}{$
          \begin{aligned}
            \mat{\lm, 1 0; 0 \lm, 1; 0 0 \lm} &= \mat{\lm, 0 0; 0 \lm, 0; 0 0 \lm} \mat{1 1/\lm, 0; 0 1 1/\lm; 0 0 1} \\
                                              &= \mat{\lm, 0 0; 0 \lm, 0; 0 0 \lm} \mat{1 \tan\phi, 0; 0 1 \tan\phi; 0 0 1}
          \end{aligned}
        $} \\
        So every Jordan block describes a combination of stretching and shearing. As for eigenvalue 0, it represents a projection (into a smaller space). As for complex eigenvalues, it can include rotations.
        \\\\
        Since every LT can be described by a collection of Jordan blocks, then \underline{every LT} is a combination of stretching, projecting, shearing, and rotating.
  \section*{\underline{Multilinear algebra}}
    \paragraph{6.1 Multilinear form} A multilinear form of degree $m$ on an $n$-dimensional vector space $V$ is a mapping
      \[ \underbrace{V \times V \times V \times \cdots}_{m \; \text{times}} \rightarrow \F \]
      which is linear in every slot.
    \paragraph{6.1A Vector space} The set of multilinear forms of degree $m$ on $V$ is a vector space in the usual way.
    \subsection*{2-forms}
      \paragraph{6.2 2-form} A bilinear form $\psi$ on a real vector space $V$ is called a 2-form if it is antisymmetric:
        \[ \psi(u, v) = -\psi(v, u) \]
        so the space of 2-forms is a subset of $B(V)$. By 6.1A, the set of 2-forms on $V$ is a vector space.
      \paragraph{6.3 Wedge product} Let $\alpha, \beta \in \hat{V}$. The wedge product of $\alpha$ and $\beta$ is defined as:
        \[ \alpha \wedge \beta = \frac{1}{2} (\alpha \otimes \beta - \beta \otimes \alpha) \]
        and it is a 2-form.
        \ull {
          \item It is indeed antisymmetric. For $u, v \in V$:
            \begin{align*}
              (\alpha \wedge \beta) (u, v) &= \frac{1}{2} \Big( \alpha(u)\beta(v) - \beta(u)\alpha(v) \Big) \\
                                         &= -\frac{1}{2} \Big( \alpha(v)\beta(u) - \beta(v)\alpha(u) \Big) \\
                                         &= - (\beta \wedge \alpha) (v, u)
            \end{align*}
          \item Hence $\alpha \wedge \alpha = -\alpha \wedge \alpha$, i.e. $\alpha \wedge \alpha = 0$.
        }
      \paragraph{6.4 Basis} Let $\zeta$ be the dual basis to the basis $z$ for $V$.
        \ull {
          \item The full set $\zeta^i \wedge \zeta^j$ is a basis for the space of 2-forms on $V$.
          \item Any 2-form $\psi$ can be written as $\psi = \psi_{ij} \zeta^i \wedge \zeta^j$.
        }
      \paragraph{2-forms on 3-dimensional vector space} For a real 3-dimensional vector space with dual basis $\zeta^1, \zeta^2, \zeta^3$, the basis vectors of 2-forms are
        \[ \zeta^1 \wedge \zeta^2, \quad \zeta^2 \wedge \zeta^3, \quad \zeta^3 \wedge \zeta^1 \]
        There are only 3 because:
        \ull {
          \item $\zeta^i \wedge \zeta^i = 0$ so they disappear
          \item $\zeta^i \wedge \zeta^j = -\zeta^j \wedge \zeta^i$ so they are linearly dependent
        }
        so the dimension of the space of 2-forms on a three dimensional space is 3. In particular, it is $C^3_2$.
    \subsection*{3-forms}
      \paragraph{Definition} A trilinear form $\psi$ on a real vector space $V$ is called a 3-form if it is antisymmetric in all 3 slots, e.g.
        \[ \psi(u,v,w) = -\psi(v,u,w) = -\psi(u,w,v) \]
        Swapping any pair of vectors will negate the result.
      \paragraph{Wedge product} Let $\alpha, \beta, \gamma \in \hat{V}$. The wedge product of $\alpha, \beta$ and $\gamma$ is defined as:
        \begin{align*}
          \alpha \wedge \beta \wedge \gamma =& \frac{1}{6} (\alpha \otimes \beta \otimes \gamma - \alpha \otimes \gamma \otimes \beta \\
                                            &+ \beta \otimes \gamma \otimes \alpha - \beta \otimes \alpha \otimes \gamma \\
                                            &+ \gamma \otimes \alpha \otimes \beta - \gamma \otimes \beta \otimes \alpha)
        \end{align*}
        If there are an odd number of swaps between vectors, then we put a minus sign. Again, this is a 3-form.
      \paragraph{Basis} Let $\zeta$ be the dual basis to the basis $z$ for $V$. Then
        \ull {
          \item The full set $\zeta^i \wedge \zeta^j \wedge \zeta^k$ is a basis for the space of 3-forms on $V$.
          \item Any 3-form $\psi$ can be written as $\psi = \psi_{ijk} \zeta^i \wedge \zeta^j \wedge \zeta^k$.
          \item The dimensionality of the space of 3-forms on an $n$-dimensional real vector space is $C^n_3$.
        }
    \subsection*{$n$-forms} \noindent
      \textcolor{red}{For this section, let $V$ be a $n$-dimensional vector space.}
      \paragraph{6.5 Dimensionality} The space of $m$-forms on $V$ is a vector space of dimension $C^n_m$.
      \paragraph{6.6} The space of $n$-forms on $V$ is a vector space of dimension $C^n_n$, i.e. it is 1-dimensional.
      \paragraph{6.6A Transpose mapping} Let $T$ be a linear operator on $V$. Then define its transpose $\hat{T}$ on $n$-forms on $V$ by
        \[ \hat{T}(\psi)(u,v,w,\cdots) = \psi(Tu,Tv,Tw,\cdots) \]
        Then $\hat{T}(\psi)$ is also a $n$-form.
        \paragraph{6.6B Repeated inputs} If a $n$-form has repeated inputs, then it evaluates to 0. \red{Because of the antisymmetry property, and we can perform a ``swap'' among the repeated inputs}
        \[ \psi(u,u,\cdots) = -\psi(u,u,\cdots) \iff 2\psi(u,u,\cdots) = 0 \]
    \subsection*{$\Delta(T)$}
      \paragraph{6.7 Definition} Let $\Omega$ be a non-zero $n$-form on $V$. Let $T$ be a linear operator on $V$. By 6.6A, $\hat{T}\Omega$ is an $n$-form. By 6.6, the space of $n$-forms on $V$ is 1-dimensional. Hence,
        \[ \hat{T}\Omega = \Delta(T) \Omega \]
        where $\Delta(T)$ is a scalar.
      \paragraph{6.8 Product} $\Delta(TS) = \Delta(T) \Delta(S)$
      \paragraph{Properties of $\Delta(T)$ for matrices}
        \oll {
          \item $\Delta(I) = 1$ \red{obvious when considering 6.6A definition}
          \item $\Delta$ is invariant under a similarity transformation, since
            \begin{align*}
              \Delta(P\inv M P) &= \Delta(P\inv) \Delta(M) \Delta(P) \\
                                &= \Delta(P\inv P) \Delta (M) = \Delta(M)
            \end{align*}
          \item Let $M$ be a order $n$ square matrix. Then
            \[ \Delta(M) = \Delta(PJP\inv) = \Delta(J) \]
            where $J$ is the Jordan canonical form.
          \item For a Jordan block of size $s$, $\Delta(J) = \lm^s$, since
            \[ \Omega(Je_i) = \Omega(\lm e_i) = \lm^s \Omega(e_i) \]
            \red{LHS is $\hat{\Omega}$ because it is the RHS of 6.6A}
          \item (6.9) Let $M$ have JCF $J$. Then
            \[ \Delta(M) = \Delta(J) = \lm_1^{m_1} \lm_2^{m_2} \cdots \]
        }
      \paragraph{Properties of $\Delta(T)$ for LTs}
        \oll {
          \item (6.10) $\Delta(T) = \Delta(\mathcal{M}(T))$
          \item (6.11) $\Delta(T) = \lm_1^{m_1} \lm_2^{m_2} \cdots$, by 6.9
          \item (6.12) $\Delta(T) \neq 0 \iff \lm \neq 0 \iff T$ is bijective.
          \item (6.13) $\Delta(T - \lm I) = 0$ \red{because $T - \lm I$ must not be bijective}
          \item $\Delta(T)$ is the determinant of $T$. It is the eigenvalue of $\hat{T}$ on a 1-dimensional vector space.
        }
      \paragraph{Proof for 6.10} Let $z: \F^n \rightarrow V$ be a basis on $V$. Let $\Omega$ be a $n$-form on $V$. Let $a,b,\cdots$ be vectors in $\F^n$. Define a non-zero $n$-form $\Omega^*$ on $\F^n$:
        \[ \Omega^*(a,b,\cdots) = \Omega(za, zb, \cdots) \]
        Let $T: V \rightarrow V$ be a LT.
        \begin{align*}
          & \Delta(T) \Omega(za, zb, \cdots) \\
          &= \hat{T} \Omega(za, zb, \cdots) & \text{by 6.7} \\
          &= \Omega(Tza, Tzb, \cdots) & \text{by 6.6A} \\
          &= \Omega^*(z\inv Tza, z\inv Tzb, \cdots) & \text{by $\Omega^*$ defn} \\
          &= \Omega^*(\mathcal{M}(T)a, \mathcal{M}(T)b, \cdots) & \text{by Fact 3.65} \\
          &= \Delta(\mathcal{M}(T)) \Omega^*(a, b, \cdots) & \text{by 6.6A and 6.7} \\
          &= \Delta(\mathcal{M}(T)) \Omega(za, zb, \cdots) & \text{by $\Omega^*$ defn}
        \end{align*}
        Since $a,b,\cdots$ are arbitrary and $\Omega \neq 0$, then
        \[ \Delta(T) = \Delta(\mathcal{M}(T)) \]
      \paragraph{Every square matrix is similar to its transpose} Let $A = PDP\inv$.
        \[ A\tr = (PDP\inv)\tr = (P\inv)\tr D\tr P\tr = (P\tr)\inv D P\tr \]
      \paragraph{Other properties}
        \ull {
          \item Because every square matrix is similar to its transpose, by property 1 for matrices,
            \[ \Delta(M) = \Delta(P\inv M P) = \Delta(M\tr) \]
          \item The determinant switches sign when swapping two rows/columns in the original matrix, because $n$-forms are antisymmetric in all pairs of slots. 
          \item We can defined a generalized parallelopiped, formed using the eigenvectors of LT $T$. Then, under $T$, its volume changes by a factor given by the product of the stretching factors in all directions, which is the product of the stretching factors for all eigenvectors, which is precisely $\lm_1^{m_1} \lm_2^{m_2} \cdots$.
        }
      \paragraph{Triple product on $\R^3$} For $u,v,w \in V$, the vector triple product is the map
        \[ (u,v,w) \rightarrow u \cdot v \times w \]
        It is trilinear and antisymmetric in all slots, so it is a 3-form on $\R^3$. Since 3 is both in the 3-form and the dimensionality of $\R^3$, then by 6.7,
        \[ (u,v,w) \rightarrow (Tu) \cdot (Tv) \times (Tw) = \Delta(T) (u \cdot v \times w) \]
        is another 3-form, which is proportional to the triple product. This was how we defined the determinant in $\R^3$ in Chapter 1.
        \\\\
        Note: the triple product on $\R^3$ is defined using the lengths of vectors and the angles between them, so it depends on the choice of $dot$ as the inner product. It is called the volume form associated with $dot$.
    \subsection*{Tensors}
      \paragraph{6.14} Let $V$ be a vector space. Let $R: V \times V \rightarrow \LVV$. $R$ is called a tensor.
\pagebreak
  \section*{\underline{Lengths and angles}} \noindent
    \paragraph{5.1 Dot product on $\R^n$} Let $u = u^i e_i$ and $v = v^j e_j$, where $u,v \in \R^n$. Then $\displaystyle u \cdot v = \sum_{k=1}^n a^k b^k$, i.e. we take the sum of the products elementwise.
      \begin{multicols}{2}
        \ull {
          \item Length: $\abs{u} = \sqrt{u \cdot u}$
          \item Components: $e_k \cdot v = v^k$
          \item Cosine of angle: $\cos(\theta) = \dfrac{u \cdot v}{\abs{u}\abs{v}}$
        }
      \end{multicols}
    \subsection*{Bilinear forms}
      \paragraph{5.2 Bilinear form} A bilinear form on a vector space $V$ is a mapping $b: V \times V \rightarrow \R$ which is linear in both slots:
        \begin{align*}
          b(cu + dv, w) &= cb(u, w) + db(v, w) \\
          b(w, cu + dv) &= cb(w, u) + db(w, v)
        \end{align*}
        for $u,v,w \in V$ and $c,d \in \R$.
        \\\\
        Note: not necessary that $b(u, v) = b(v, u)$.
      \paragraph{5.4} The set of bilinear forms on $V$, $B(V)$, is a vector space in the usual way.
      \paragraph{5.5 Tensor product} Let $\alpha, \beta \in \hat{V}$, and let $u, v \in V$. Then the tensor product of $\alpha$ with $\beta$ is the element of $B(V)$ such that
        \[ \alpha \otimes \beta (u,v) = \alpha(u) \beta(v) \]
      \paragraph{5.8, 5.9 Basis for $B(V)$} Let $z_i$ be a basis for $V$ and let $\zeta^j$ be the dual basis.
        \ull {
          \item For each $(i, j)$, $\zeta^i \otimes \zeta^j \in B(V)$.
          \item The full set $\zeta^i \otimes \zeta^j$ over all $(i, j)$ is a basis for $B(V)$.
          \item The dot product on $\R^n$ is an element of $B(V)$.
        }
      \paragraph{Proof for span}
        \oll {
          \item Let $b \in B(V)$ and define $b_{ij} = b(z_i, z_j)$.
          \item Define $h = b_{ij} \zeta^i \otimes \zeta^j$, and let it act on $(z_k, z_l)$:
            \begin{align*}
              h(z_k, z_l) &= b_{ij} \zeta^i \otimes \zeta^j (z_k, z_l) = b_{ij} \zeta^i(z_k) \zeta^j(z_l) \\
                          &= b_{ij} I^i_k I^j_l = b_{kl} = b(z_k, z_l)
            \end{align*}
            By linearity, then $h(u, v) = b(u, v)$ for all $u, v \in V$.
          \item Thus, $h = b$, that is $b = b_{ij} \zeta^i \otimes \zeta^j$, so the full set of $\zeta^i \otimes \zeta^j$ span $B(V)$. We call the numbers $b_{ij}$ the components of $b$ relative to basis $z_i$.
        }
      \paragraph{5.11 Extracting components} \noindent \\ Since $b_{ij} = b(z_i, z_j)$, we just let $b$ act on $(z_i, z_j)$ to get the components of $b$.
      \paragraph{5.13} Since dot is a bilinear form, we can express it as a linear combination of $\zeta^i \otimes \zeta^j$, by first finding its components (relative to the chosen basis $e_i$):
        \[ b_{ij} = dot(e_i, e_j) = e_i \cdot e_j = I_{ij} \]
        and by observation, $dot = I_{ij} \epsilon^i \otimes \epsilon^j$. Note that we must balance the subscripts and superscripts.
      \paragraph{5.14 Matrix of bilinear form} The matrix of a bilinear form $b \in B(V)$, where $\dim(V) = 3$, rel. to basis $z_i$ is:
        \[ \mat{b_{11} b_{12} b_{13}; b_{21} b_{22} b_{23}; b_{31} b_{32} b_{33}} \]
      \paragraph{5.16 Change of basis} From Tut 6 Q7, if we change to a new vector basis $y_i = P^j_i z_j$, then the corresponding dual bases are related by $\eta^j = (P\inv)^j_i \zeta^i$, so $\zeta^j = P^j_i \eta^i$. Thus,
        \[ b = b_{ij} \zeta^i \otimes \zeta^j = b_{ij} P^i_k \eta^k \otimes P^j_h \eta^h = (P^i_k b_{ij} P^j_h) \eta^k \otimes \eta^h \]
        Refer to 5.77 to find out more
    \subsection*{Inner products}
      \paragraph{5.18 Inner product space} is a pair $(V, g)$, with a vector space $V$, and a particular inner product $g$, which is a bilinear form with the following properties:
        \oll {
          \item Positivity: $g(u, u) \geq 0 \quad \forall u \in V$
          \item Definiteness: $g(u, u) = 0 \iff u = 0$
          \item Symmetry: $g(u, v) = g(v, u) \quad \forall u,v \in V$
        }
      \paragraph{5.22} The set of inner products on $V$ is a subset of $B(V)$. But it is not a subspace, because it does not contain the zero bilinear form, as it does not satisfy definiteness.
      \paragraph{5.24, 5.25} Consider a bilinear form $b$. Consider its matrix $b_{ij}$ relative to basis $z_i$. If $b_{ij}$ is diagonal, and its diagonal entries $b_{ii} > 0$, then
        \[ b = b_{ij} \zeta^i \otimes \zeta^j \]
        is an inner product, where $\zeta^i$ is the dual basis to $z_i$.
      \paragraph{5.26 Properties of inner product matrix} Let $u^i \in \R^n$. The matrix $g_{ij}$ of the bilinear form $g$ rel. to basis $z_i$ is $(n \times n)$ matrix with the following analogous properties:
        \oll {
          \item Positivity: $g_{ij} u^i u^j \geq 0$
          \item Definiteness: $g_{ij} u^i u^j = 0 \iff u^i = 0 \quad \forall i$
          \item Symmetry: $g_{ij} = g_{ji} \quad \forall i,j$
        }
      \paragraph{Definiteness implies non-singular}
        \ull {
          \item By definition, a singular matrix $g$ has $g_{ij} u^j = 0$ where $u \neq 0$.
          \item Multiplying by $u^i$ and summing, we have $g_{ij} u^i u^j = 0$.
          \item By definiteness, the above equation implies $u^i = 0$ for all $i$, i.e. $u = 0$, contradicting our assumption.
          \item Hence $g_{ij}$ is non-singular.
        }
    \subsection*{Lengths and angles} \noindent
      \textcolor{red}{For this section, let $(V,g)$ be an inner product space.}
      \paragraph{5.29 Length} The length (or norm) of a vector $v \in V$ is
        \[ \abs{v} = \sqrt{g(v,v)} \]
      \paragraph{5.30 Orthogonal} Vectors $u,v \in V$ are orthogonal if $g(u,v) = 0$.
      \paragraph{5.31} If $u,v$ are orthogonal, then $\abs{u+v}^2 = \abs{u}^2 + \abs{v}^2$.
      \paragraph{5.33 Orthogonal decomposition} Let $u,v \in V$ where $v \neq 0$. Then $u = u_{\parallel v} + u_{\perp v}$, where
        \[ u_{\parallel v} = \frac{g(u,v)}{\abs{v}^2} v \caseand u_{\perp v} = u - \frac{g(u,v)}{\abs{v}^2} v \]
      \paragraph{5.34 Cauchy-Schwarz inequality} If $u,v \in V$, then
        \[ \abs{g(u,v)} \leq \abs{u}\abs{v} \]
        Equality occurs if and only if $u = kv$ for scalar $k$.
      \paragraph{Proof}
        \oll {
          \item If $v=0$, $g(u,0) = 0$, and RHS = 0, so the statement is true.
          \item Consider $v \neq 0$, applying 5.31 to the orthogonal decomposition:
            \begin{equation}\label{5.34.1}
              \abs{u}^2 = \abs{u_{\parallel v}}^2 + \abs{u_{\perp v}}^2
            \end{equation}
          \item Since (\ref{5.34.1}) is a sum of squares, we have $\abs{u}^2 \geq \abs{u_{\parallel v}}^2$.
          \item Then, compute $u_{\parallel v}^2$: \\
            \resizebox{0.9\hsize}{!}{$
              \begin{aligned}
                \abs{u_{\parallel v}}^2 &= g\left( \frac{g(u,v)}{\abs{v}^2} v, \frac{g(u,v)}{\abs{v}^2} v \right) = \left( \frac{g(u,v)}{\abs{v}^2} \right)^2 g(v,v) \\
                                        &= \left( \frac{g(u,v)}{\abs{v}^2} \right)^2 \abs{v}^2 = \frac{g(u,v)^2}{\abs{v}^2}
              \end{aligned}
            $}
          \item Susbtituting into the inequality $\abs{u}^2 \geq \abs{u_{\parallel v}}^2$, we get
            \[ \abs{u}^2 \geq \frac{g(u,v)^2}{v^2} \]
            Rearranging and taking the positive square root, we get the Cauchy-Schwarz inequality.
          \item Referring to (\ref{5.34.1}), we get equality if $\abs{u_{\perp v}} = 0$, i.e. $u_{\perp v} = 0$. That is, $u$ and $v$ are parallel.
        }
      \paragraph{5.35 Triangle inequality} For $u,v \in V$,
        \[ \abs{u+v} \leq \abs{u} + \abs{v} \]
        Equality occurs if and only if $u = kv$ for $k \geq 0$.
      \paragraph{Proof}
        Again for $u=0$ it is obvious. For $u \neq 0$, \\\\
        \resizebox{\hsize}{!}{$
          \begin{aligned}
              \abs{u+v}^2 = \abs{u}^2 + \abs{v}^2 + 2g(u,v) &\leq \abs{u}^2 + \abs{v}^2 + 2\abs{g(u,v)} \\
                                                            &\leq \abs{u}^2 + \abs{v}^2 + 2\abs{u}\abs{v} \\
                                                            &= (\abs{u} + \abs{v})^2
          \end{aligned}
        $}
        For equality, we must first get equality from the Cauchy-Schwarz inequality, i.e. $v = ku$ for some scalar $k$. Also, we must satisfy $\abs{g(u,v)} = \abs{u}\abs{v}$. Substituting, we must have $k \geq 0$.
      \paragraph{5.36 Angle} Let $u,v \in V$. The angle between $u$ and $v$, $\theta$ is
        \[ \cos(\theta) = \frac{g(u,v)}{\abs{u}\abs{v}} \]
        where $0 \leq \theta \leq \pi$. Note that this equation makes sense because of Cauchy-Schwarz, which ensures that the RHS lies in $[-1, 1]$.
    \subsection*{Orthonormal bases} \noindent
      \textcolor{red}{For this section, let $(V,g)$ be an inner product space.}
      \paragraph{5.37, 5.38 Orthonormal basis} Let $z_i$ be a basis on $V$. It is an orthonormal basis if for all $i, j$, we have
        \[ g(z_i, z_j) = I_{ij} \]
        or equivalently, its matrix relative to $z_i$ is $I$.
      \paragraph{5.40} The canonical basis for $\R^n$ is an orthonormal basis for $(\R^n, \cdot)$.
      \paragraph{5.41} We can alternatively define an orthonormal basis $z$ as follows:
        \[ g(za, zb) = dot(a, b) \]
        where $a,b \in \R^n$.
      \paragraph{5.42} Suppose $y_i$ is a list of orthornormal vectors. Then
        \[ \abs{a^i y_i}^2 = (a^1)^2 + (a^2)^2 + (a^3)^2 + \cdots \]
      \paragraph{5.43 Orthonormal vectors are LI} Any list of orthonormal vectors is LI. \red{square root both sides of 5.42, then equate to 0. Sum of squares is 0 $\Rightarrow$ each term is 0}
      \paragraph{5.44} Any list of $n = \dim(V)$ orthonormal vectors is a basis for $V$.
    \subsection*{Gram-Schmidt orthogonalization}
      \paragraph{5.45 Unit vector} A unit vector $v^+$ is one where $\abs{v^+} = 1$. For any vector $u$, we can scale it to construct a unit vector:
        \[ u^+ = \frac{u}{\abs{u}} \]
      \paragraph{5.46 Parallel component of unit vector}
        If $v$ is a unit vector, then for any vector $u$,
        \[ u_{\parallel v} = \frac{g(u,v)}{\abs{v}^2} v = g(u,v) v \]
      \paragraph{5.46A GS} is an algorithm that transforms a basis into an orthonormal basis. Example for $\dim(V) = 3$:
        \begin{align*}
          z_1^+ &= \frac{z_1}{\abs{z_1}} \\
          z_2^+ &= \frac{z_2 - g(z_2, z_1^+) z_1^+}{\abs{z_2 - g(z_2, z_1^+) z_1^+}} \\
          z_3^+ &= \frac{z_3 - g(z_3, z_1^+) z_1^+ - g(z_3, z_2^+) z_2^+}{\abs{z_3 - g(z_3, z_1^+) z_1^+ - g(z_3, z_2^+) z_2^+}}
        \end{align*}
        \ull {
          \item Removes the parallel component of each new orthonormal basis vector.
          \item Each $z_i^+$ is a linear combination of $z_1, z_2, \cdots, z_i$, so the change-of-basis matrix is UT.
        }
      \paragraph{5.47} Every inner product space has an orthonormal basis.
    \subsection*{Riesz representation} \noindent
      \textcolor{red}{For this section, let $(V,g)$ be an inner product space.}
      \paragraph{5.48 Riesz representation} Let $u \in V$. Define a dual vector $\Gamma_u$:
        \[ \Gamma_u(v) = g(v,u) \quad \forall v \in V \]
        It maps $V$ to $\R$ and is linear because $g$ is bilinear. Hence $\Gamma_u \in \hat{V}$.
      \paragraph{5.49} Define a mapping $\Gamma: V \rightarrow \hat{V}$ such that
        \[ \Gamma: u \rightarrow \Gamma_u \]
        $\Gamma$ is linear because $g$ is bilinear. $\Gamma$ is an isomorphism.
      \paragraph{Proof (Isomoprhism)}
        \oll {
          \item Assume $\Gamma$ is not injective. Then some $\Gamma(v)$ for $v \neq 0$ is the zero dual vector.
          \item Then $\Gamma(v)(u) = 0 \quad \forall u \in V$.
          \item In particular, $\Gamma(v)(v) = 0$ for some $v \neq 0$, contradicting the definiteness of $g$, so $\Gamma$ is injective.
          \item By FT of LA, the $Rank(\Gamma) = \dim(V)$. But we know $\dim(V) = \dim(\hat{V})$.
          \item Thus $Rank(\Gamma) = \dim(\hat{V})$ so the range of $\Gamma$ is all of $\hat{V}$, so $\Gamma$ is surjective.
          \item Since $\Gamma$ is linear (since $g$ is linear in both slots), then $\Gamma$ is an isomorphism.
        }
      \paragraph{5.50 Riesz representation} Any dual vector $\alpha$ can be expressed as $\Gamma(u) = \Gamma_u$ for some unique $u \in V$.
        \[ \alpha(v) = \Gamma(u)(v) = g(v,u) \quad \forall v \in V \]
      \paragraph{5.51 Dual basis of orthonormal basis} For an orthonormal basis $z_i$, the dual basis is $\Gamma(z_j)$.
      \paragraph{Proof}
        \ull {
          \item Let $v = a^iz_i \in V$.
          \item By 5.50, $\Gamma(z_j)(z_i) = g(z_i, z_j) = I_{ij}$
          \item $\Gamma(z_j)v = \Gamma(z_j)(a^iz_i) = a^i \Gamma(z_j)(z_i) = I_{ij}a^i = a^j$
          \item The $j$th component of $v$ is obtained by letting $\Gamma(z_j)$ act on $v$, i.e. it is a dual vector.
        }
      \paragraph{Riesz representation in terms of components}
        \ull {
          \item Let $z_i$ be a basis, let $\zeta^j$ be the dual basis.
          \item Let $\alpha = p_i \zeta^i, u = a^j z_j, v = b^k z_k$.
          \item Then $\alpha(v) = g(v,u)$ is written as
            \[ p_i \zeta^i b^k \zeta_k = p_k b^k = g(b^k z_k, a^j z_j) = a^j b^k g_{kj} \]
          \item Since this is true for arbitrary $v$ (and thus arbitrary $b^k$, we must have $p_k = g_{kj} a^j$.
          \item This means that $u$ can be found from $\alpha$, and it asserts that matrix $g_{kj}$ is not singular.
        }
    \subsection*{The complex case}
      \paragraph{Dot product on $\C^n$} Let $u = u^i e_i$ and $v = v^j e_j$, where $u,v \in \C^n$. Then $\displaystyle u \cdot v = \sum_{k=1}^n a^k \bar{b}^k$.
      \paragraph{Properties of complex conjugates} Let $z = a + bi \in \C$. Let $w \in \C$.
        \begin{multicols}{2}
          \ull {
            \item $\bar{z} = \overline{a+bi} = a - bi$
            \item $\overline{z \pm w} = \bar{z} \pm \bar{w}$
            \item $\overline{(zw)} = \bar{z} \; \bar{w}$
            \item $\overline{(\bar{z})} = z$
            \item $\overline{(\frac{z}{w})} = \frac{\bar{z}}{\bar{w}}$
            \item $z$ is real $\iff \bar{z} = z$.
          }
        \end{multicols}
      \paragraph{Conjugate dual vectors (Tut 10 Q5)} Let $V$ be a complex vector space. Let $\gamma: V \rightarrow \C$ be a mapping which is conjugate linear, that is if $a,b \in \C$ and $u,v \in V$, then $\gamma$ satisfies
        \[ \gamma(au + bv) = \bar{a} \gamma(u) + \bar{b} \gamma(v) \]
        Also, if $\alpha$ is a dual vector and $v \in V$, then if we define $\bar{\alpha}$ as follows, it is a conjugate dual vector:
        \[ \bar{\alpha}(v) = \overline{\alpha(v)} \]
      \paragraph{5.53 Sesquilinear form} A sesquilinear form on a complex vector space $V$ is a mapping $s: V \times V \rightarrow \C$ which is linear in the first slot, but conjugate-linear in the second: 
        \begin{align*}
          s(cu, v) &= cs(u, v) \\
          s(u, cv) &= \bar{c}s(u, v)
        \end{align*}
        for $u,v \in V$ and $c \in \C$.
      \paragraph{5.54 Complex inner product space} is a pair $(V, g)$, with a vector space $V$, and a particular sesquilinear form $g$ that satisifes
        \oll {
          \item Positivity: $g(u, u) \geq 0 \quad \forall u \in V$
          \item Definiteness: $g(u, u) = 0 \iff u = 0$
          \item Symmetry: $g(u, v) = \overline{g(v, u)} \quad \forall u,v \in V$
        }
      \paragraph{Other properties}
        \ull {
          \item Matrix is same as in 5.14
          \item $g(u, u) \in \R$ by the symmetry property
          \item Length is same as in 5.29 \red{since $g(u, u) \in \R$}
          \item Orthogonality, Cauchy-Schwarz is same as in 5.30, 5.34
        }
      \paragraph{Triangle inequality} For $u,v \in V$,
        \[ \abs{u+v}^2 \leq \abs{u}^2 + \abs{v}^2 + 2\abs{u}\abs{v} \]
      \paragraph{Proof} Again for $u=0$ it is obvious. For $u \neq 0$: \\\\
        \resizebox{\hsize}{!}{$
          \begin{aligned}
            \abs{u+v}^2 &= g(u+v, u+v) = \abs{u}^2 + \abs{v}^2 + g(u,v) + g(v,u) \\
                        &= \abs{u}^2 + \abs{v}^2 + g(u,v) + \overline{g(u,v)} \\
                        &= \abs{u}^2 + \abs{v}^2 + 2 \mathrm{Re}(g(u,v)) \\
                        &\leq \abs{u}^2 + \abs{v}^2 + 2 \abs{g(u,v)} \quad \text{since $\mathrm{Re}(c) \leq \abs{c}$} \\
                        &\leq \abs{u}^2 + \abs{v}^2 + 2 \abs{u}\abs{v} \quad \text{by Cauchy-Schwarz}
          \end{aligned}
        $}
      \paragraph{5.55 Riesz representation} Let $u \in V$. Define a dual vector $\Gamma_u$:
        \[ \Gamma_u(v) = g(v,u) \quad \forall v \in V \]
        In the complex case, it is not an isomorphism. Instead, it is a conjugate isomorphism.
      \paragraph{5.56 Conjugate isomorphism} Let $V,W$ be complex vector spaces. A conjugate isomorphism $F: V \rightarrow W$ is a bijective map that satisfies
        \[ F(au + bv) = \bar{a} F(u) + \bar{b} F(v) \quad \forall a,b \in C, u,v \in V \]
      \paragraph{5.57 Schur's theorem} Given a linear operator $T$ on a complex inner product space, there is a orthonormal basis wrt which $T$ has a UT matrix.
      \paragraph{Proof}
        \oll {
          \item By 4.26, every complex matrix is similar to a UT matrix.
          \item We can use GS to get to an orthonormal basis. By 5.46A, the change-of-basis matrix is UT.
          \item By Tut 9 Q5, the product of two UT matrices is UT, hence proven.
        }
    \subsection*{Spectral theorem}
      \paragraph{5.58 Bijection between LTs and \\ sesquilinear/bilinear forms} \noindent
        \\\\
        By the Riesz representation, we can think of vectors as dual vectors. We can also think of LTs as sesquilinear forms as follows. (The same explanation applies if we define $\tau$ to be bilinear)
        \\\\
        Let $T$ be an operator on the complex inner product space $(V,g)$. Then define a sesquilinear form $\tau$:
        \[ \tau(u, v) = g(u, Tv) \quad \forall u,v \in V \]
        To get the components,
        \[ \tau_{ij} = \tau(z_i, z_j) = g(z_i, Tz_j) = g(z_i, T^k_j z_k) = g_{ik} \overline{T^k_j} \]
        Since $g_{ij}$ is invertible, we can express
        \[ \overline{T^k_j} = g^{ik} \tau_{ij} \]
        where $g^{ik}$ is the inverse of $g_{ik}$, so that each $\tau$ defines a unique operator on $(V,g)$.
      \paragraph{5.59 Hermitian/Symmetric form} A sesquilinear/bilinear form is Hermitian (in the complex case) or symmetric (in the real case) if
        \[ \tau(u,v) = \overline{\tau(v,u)} \quad \forall u,v \in V \]
      \paragraph{5.60 Hermitian/Symmetric LT} A LT on an inner product space is Hermitian/symmetric if its corresponding sesquilinear/bilinear form is Hermitian/symmetric. In particular, $T$ satisfies
        \[ g(u, Tv) = \textcolor{red}{\tau(u, v) = \overline{\tau(v, u)}} = \overline{g(v, Tu)} = g(Tu, v) \]
        and the matrix of $\tau$ is Hermitian/symmetric:
        \[ \tau_{ij} = \textcolor{red}{\tau(z_i, z_j) = \overline{\tau(z_j, z_i)}} = \overline{\tau_{ji}} \]
        in the complex case, and $\tau_{ij} = \tau_{ji}$ in the real case.
      \paragraph{5.63} If we use an orthonormal basis, then the matrices of $T$ and $\tau$ are complex conjugates of each other. In particular, if they are real, then they are equal.
        \[ \tau_{ij} = g_{ik} \overline{T^k_j} = \overline{T^i_j} \]
        because $g_{ik} = 1 \iff i = k$.
      \paragraph{5.64} The eigenvalues of a Hermitian/symmetric LT are real.
      \paragraph{5.65 Proof} By 4.21, let $T$ have an eigenvector $u$ with eiegenvalue $\lm$. On one hand we have
        \begin{equation}\label{5.64.1}
          g(u, Tu) = g(u, \lm u) = \bar{\lm} g(u, u)
        \end{equation}
        on the other hand
        \begin{equation}\label{5.64.2}
          g(Tu, u) = \lm g(u, u)
        \end{equation}
        By the definition in 5.60, LHS of (\ref{5.64.1}) and (\ref{5.64.2}) are equal. Thus, we have
        \[ \bar{\lm} g(u, u) = \lm g(u, u) \]
        but since $u \neq 0 \Rightarrow g(u, u) \neq 0$ (definiteness), then
        \[ \bar{\lm} = \lm \]
        i.e. $\lm$ is real.
      \paragraph{5.66, 5.67 Spectral theorem} Any Hermitian LT $T$ on a complex inner product space has a real, diagonal matrix relative to some orthonormal basis, i.e. it can be diagonalized and the resulting matrix is real.
      \paragraph{Proof}
        \oll {
          \item By 5.57, $T$ can be made UT wrt an orthonormal basis.
          \item By 5.63, the associated sesquilinear form is also UT.
          \item By 5.61, $\tau_{ij} = \overline{\tau_{ji}}$. The UT entries are the complex conjugates of the lower-triangular (LT) entries.
          \item Since the LT entries are 0, the UT entries are 0.
          \item Hence the matrix is diagonal.
        }
      \paragraph{5.68} The spectral theorem also applies to symmetric LTs on real vector spaces.
      \paragraph{5.69} The eigenvectors of a Hermitian transformation form an orthonormal basis. \red{The basis used is orthonormal. Since it can diagonalize the matrix, it is a basis of eigenvectors.}
      \paragraph{Tut 10 Q1} The eigenvectors of an Hermitian transformation corresponding to distinct eigenvalues are orthogonal. \red{consider $g(u,Tv) = g(Tu,v)$, letting $u$ and $v$ be eigenvectors with distinct eigenvalues}
      \paragraph{5.70 Spectral decomposition} Let $S$ be a real symmetric bilinear form, let $z_i$ be the orthonormal basis of eigenvectors mentioned in 5.69, let $\lm_i$ be the eigenvalues, and let $\zeta^i$ be the dual basis to $z_i$. Since the matrix relative to the basis is diagonal, then we have
        \[ S = \lm_1 \zeta^1 \otimes \zeta^1 + \lm_2 \zeta^2 \otimes \zeta^2 + \cdots \]
    \subsection*{Decomposing matrices}
      \paragraph{5.71 Transpose matrix} The transpose of a (not necessarily square) matrix $M = M_{ij}$ is $M \tr = M_{ji}$.
      \paragraph{5.72 Orthogonal matrix} A real matrix $P$ is orthogonal if
        \oll {
          \item $P \tr P = I$, or
          \item Rows/columns are all unit vectors, and rows/columns are orthogonal to other rows/columns wrt $dot$ on $\R^n$.
        }
      \paragraph{5.76} Any symmetric matrix $S$ can be expressed in the form $O\tr D O$, where $D$ is diagonal and real, and $O$ is orthogonal. We say that all symmetric real matrices are orthogonally diagonalizable.
      \paragraph{Unitary matrix} A complex matrix is unitary if
        \oll {
          \item $\overline{U\tr} U = I$, or
          \item Rows/columns are all unit vectors, and rows/columns are orthogonal to other rows/columns wrt $dot$ on $\C^n$.
        }
      \paragraph{5.77} Any Hermitian matrix $H$ can be expressed in the form $\overline{U\tr} D U$, where $D$ is diagonal and real, and $U$ is a complex unitary matrix. We say that all Hermitian matrices are unitarily diagonalizable.
      \paragraph{Example} Let $S$ be a real, symmetric matrix.
        \oll {
          \item Consider the inner product space $(\R^n, dot)$, which has canonical dual basis $\epsilon^i$. By 5.24, we can define a bilinear form on $\R^n$. By 5.60, if the bilinear form $S$ is symmetric, then the LT $S$ is also symmetric. By 5.58, we have a bijection between bilinear forms and LTs so the statement of 5.60 is an if and only if. So indeed the bilinear form $S$ is symmetric.
            \[ S = S_{ij} \epsilon^i \otimes \epsilon^j \]
          \item Next, since we have a real symmetric bilinear form, by 5.70, we know that $\R^n$ has an orthonormal basis $z_i$ wrt inner product $dot$, wrt which the matrix of $S$ is a diagonal real matrix:
            \[ S = D_{ii} \zeta^i \otimes \zeta^i = D_{ij} \zeta^i \otimes \zeta^j \]
            \ull {
              \item RHS is still equal because $D$ is a diagonal matrix.
              \item $z_i$ is not the canonical basis $e_i$, because if it were, then the original matrix would have been diagonal.
            }
          \item Let $P^i_j$ be the change-of-basis matrix from $z_i$ to the canonical basis $e_i$, so $e_i = P^j_i z_j$. By 5.16, we have
            \[ S = (P^h_i D_{hk} P^k_j) \epsilon^i \otimes \epsilon^j \]
            so $S_{ij} = P^h_i D_{hk} P^k_j$.
          \item In terms of matrices, by 5.71, we have $S = P\tr D P$ for some matrix $P$.
          \item Then, since $z_i$ and $e_i$ are both orthonormal bases wrt $dot$, and since $e_i = P^j_i z_j$, we have
            \[ dot(e_i, e_j) = I_{ij} = dot(P^h_i z_h, P^k_j z_k) = P^h_i I_{hk} P^k_j \]
            i.e. $P\tr P = I$, so $P$ is orthogonal.
        }
\end{multicols*}
\end{document}
