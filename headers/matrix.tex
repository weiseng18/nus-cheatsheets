\usepackage{spalign}

\let\mat=\spalignmat
\let\amat=\spalignaugmat

\newcommand{\detmat}[1]{\spaligndelims\vert\vert\spalignmat{#1}\spaligndelims()}

\newcommand{\ro}[1]{\xrightarrow{\begin{subarray}{l}#1\end{subarray}}}
\newcommand{\roTwo}[2]{\xrightarrow{\begin{subarray}{l}#1\\#2\end{subarray}}}
\newcommand{\roThree}[3]{\xrightarrow{\begin{subarray}{l}#1\\#2\\#3\end{subarray}}}

% matrix operations
\newcommand{\rref}{\mathrm{rref}}

\newcommand{\transpose}{^{\text{T}}}
\newcommand{\tr}{^{\text{T}}}

\newcommand{\GJE}{\ro{\text{GJE}}}
\newcommand{\matdim}[1]{\mathrm{dim}(#1)}
\newcommand{\matrank}[1]{\mathrm{rank}(#1)}
\newcommand{\matnullity}[1]{\mathrm{nullity}(#1)}

\newcommand{\minrank}[1]{\mathrm{min}\{#1\}}

\newcommand{\vabs}[1]{\abs{\abs{#1}}}

\newcommand{\by}[2]{$#1 \times #2$}
\newcommand{\laspan}[1]{\mathrm{span}\left\{#1\right\}}

\newcommand{\spc}{\spaligndelims..} % spalign cancel brackets
\newcommand{\spv}{\spaligndelims\vert\vert} % spalign vert
\newcommand{\spr}{\spaligndelims()} % spalign reset

% vector bold
\newcommand{\vb}[1]{\boldsymbol{#1}}
\newcommand{\I}{\vb{I}}
\newcommand{\vz}{\vb{0}}

% expand to row vector
% #1 is the vector, #2 is the length
% refer to MA2001 Tutorial 5 for sample usage
\newcommand{\exprow}[2]{
  \rule[.4ex]{#2}{0.5pt}
  \hspace{0.2cm}
  #1
  \hspace{0.2cm}
  \rule[.4ex]{#2}{0.5pt}
}

% set of column vectors
\newcommand{\vset}[1]{\left\{#1\right\}}
